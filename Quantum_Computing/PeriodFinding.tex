\subsection{Period Finding}
\label{subsec: Period Finding}

\textbf{Problem:} Suppose $f$ is a periodic function producing a single bit as output and such that $f(x+r) = f(x)$, for some unknown $r \in (0, 2^L)$, where $x,r \in \{ 0,1,2, \dots \}$.

Given a quantum blackbox $U$, which perform $U\ket{x} \ket{y} \rightarrow \ket{x} \ket{y \oplus x} $, how many black box queries and other operations are needed to determine $r$? Here we present and algorithm to solve with only one query and $O(L^2)$ other operations.

\begin{algorithm}[H]
\SetAlgoLined
\SetKwInOut{Input}{input}
\SetKwInOut{Output}{output}
\SetKwInOut{Runtime}{runtime}
\Input{\begin{enumerate}
    \item A blackbox which perform $U\ket{x}\ket{y} = \ket{x}\ket{x + f(x)}$
    \item A state that store the function evaluation, initialized at $\ket{0}$
    \item $t = O( L + \log (\frac{1}{\epsilon})$ qubits initialized at $\ket{0}$
\end{enumerate}
}
\Output{The least integer $r>0$ such that $f(x+r)=f(x)$}
\Runtime{One use of $U$, and $O(L^2)$ operations. Succeeds with probability of $O(1)$.}

\begin{enumerate}
    \item $\ket{0}\ket{0} \rightarrow$ initial state.
    \item Create superposition: 
    \begin{equation*}
        \frac{1}{\sqrt{2^t}} \sum_{x=0}^{2^t-1} \ket{x} \ket{0}
    \end{equation*}
    \item Apply U:
    \begin{equation*}
        \frac{1}{\sqrt{2^t}} \sum_{x=0}^{2^t-1} \ket{x} \ket{f(x)} \approx \frac{1}{\sqrt{r 2^t}} \sum_{l=0}^{r-1} \sum_{x=0}^{2^t-1} e^{\frac{2 \pi i l x}{r}} \ket{x} \ket{\hat{f}(l)}
    \end{equation*}
    \item Apply reverse Fourier Transform to 1st register:
    \begin{equation*}
        \frac{1}{\sqrt{r}} \sum_{l=0}^{r-1} \ket{\Tilde{l/r}} \ket{\hat{f}(l)}
    \end{equation*}
    \item Measure first register to get $\Tilde{l/r}$.
    \item Apply continuous fraction to get $r$.
\end{enumerate}
\caption{Period finding algorithm}
\end{algorithm}

The key for this algorithm is step \textbf{3}:

\begin{equation*}
    \ket{\hat{f}(l)} = \frac{1}{\sqrt{r}} \sum_{x=0}^{r-1} e^{\frac{-2 \pi i l x}{r}} \ket{f(x)}
\end{equation*}

Where $\ket{\Tilde{f}(x)}$ is the fourier transform of $\ket{f(x)}$. This is based on:

\begin{equation*}
    \ket{f(x)} = \frac{1}{\sqrt{r}} \sum_{l=0}^{r-1} e^{\frac{2 \pi i l x}{r}} \ket{\Tilde{f}(l)}
\end{equation*}

Where 
\begin{equation*}
    \sum_{l=0}^{r-1} e^{\frac{2 \pi i l x}{r}} = \begin{cases}
    r,& \text{if } x= kr \ , \ k \in \mathbb{Z}\\
    0,              & \text{otherwise}
\end{cases}
\end{equation*}