\subsection{Quantum Counting}
\label{subsec: Quantum Counting}

Now we want to find how many solutions, M, are on an N item search problem. For a classical computer we would need $\Omega(N)$ consultations to estimate M.

That problem has some important applications: Firstly, it is important to know M on the search problem, because if $M \geq \frac{N}{2}$ we can only sample it randomly to find the solution. Secondly, this method allows us to know if there is a solution at all for the $N$ item search problem.

Quantum counting is an application of phase estimation to estimate the eigenvalues of the Grover operator, which enables us to find the number of solutions to the search problem.

\subsubsection{Algorithm}
\label{Subsubsec: Quantum Counting: Algorithm}

Suppose $\ket{a}$ and $\ket{b}$ are two eigenvectors in the space spanned y $\ket{\alpha}$ and $\ket{\beta}$. Let $\theta$ be the angle of the rotation determined by the Grover operator, which is:

\begin{equation*}
    G = \begin{bmatrix}
    \cos \theta & -\sin \theta \\
    \sin \theta & \cos \theta
    \end{bmatrix} \ \ \ \text{with} \ \sin \theta = \frac{2 \sqrt{M(N-M)}}{N}
\end{equation*}

The eigenvalues of $G$ are $e^{i\theta}$ and $e^{i(2 \pi - \theta)}$. We can assume that $M \leq \frac{N}{2}$ because we can use the augmented oracle, ensuring that $\sin^2 \frac{\theta}{2} = \frac{M}{2N}$. The circuit for quantum counting is the following:

\begin{figure}[H]
    \centering
    \begin{quantikz}
      \lstick[wires=3]{$t$}& \lstick{$\ket{0}$} & \gate[wires=3]{H^{\otimes t}}  & \qw & \qw & \qw & \ & \ & \ctrl{3} & \gate[wires=3]{FT^{\dagger}} & \meter{} \\
    & \lstick{$\ket{0}$} &  & \qw & \ctrl{2} & \qw & \  \dots \ & \ & \qw & & \meter{}  \\
    & \lstick{$\ket{0}$} &  & \ctrl{1} & \qw & \qw & \  & \ & \qw & & \meter{} \\
    \lstick[wires=2]{$n+1$}& \lstick{$\ket{0}$} & \gate[wires=2]{H^{\otimes (n+1)}} & \gate[wires=2]{G^{2^0}} & \gate[wires=2]{G^{2^1}} & \qw & \ \dots &  & \gate[wires=2]{G^{2^{t-1}}} & \qw \\
    & \lstick{$\ket{0}$} & \qw & &  & \qw & \  \dots &  & & \qw \\
    \end{quantikz}
    \caption{Quantum Counting Circuit.}
    \label{fig: Quantum Counting}
\end{figure}

The circuit estimates $\theta$ with $m$ bits of accuracy, with a probability of success of at least $1 - \epsilon$. The first register contains $t = m + \lceil \log \big( 2 + \frac{1}{2 \epsilon} \big) \rceil$ qubits, just as the phase estimation algorithm, and the second register has $n+1$ qubits to implement the augmented Grover operator. The second register is initialized at $ \ket{\psi} = \sum_x \ket{x}$, this is a superposition of $\ket{a}$ and $\ket{b}$ thus this gives us an estimation of $\theta$ or $2 \pi - \theta$ accurately with $| \Delta \theta | \leq 2^{-m}$, with probability at least $1 - \epsilon$. The estimate of $2 \pi - \theta$ is equivalent to $\theta$, so we determine $\theta$, so we certainly determine $\theta$ to an accuracy $2^{-m}$ with probability $1 - \epsilon$.

Using that $\sin^2 \big( \frac{\theta}{2} \big) = \frac{M}{2N}$ and our estimate of $\theta$ we can find M. How large is the error in $M$, that we will denote as $\Delta M$?

\begin{equation*}
    \frac{| \Delta M |}{2N} = \bigg|\sin^2 \bigg( \frac{\theta + \Delta \theta}{2}  \bigg) - \sin^2 \bigg( \frac{\theta}{2} \bigg) \bigg| = \bigg| \bigg( \sin \bigg( \frac{\theta + \Delta \theta}{2} \bigg) + \sin \bigg( \frac{\theta}{2} \bigg) \bigg) \bigg( \sin \bigg( \frac{\theta + \Delta \theta}{2} \bigg) - \sin \bigg( \frac{\theta}{2} \bigg) \bigg) \bigg|
\end{equation*}

But we know that:

\begin{equation*}
    \bigg|  \sin \bigg( \frac{\theta + \Delta \theta}{2} \bigg) + \sin \bigg( \frac{\theta}{2} \bigg) \bigg| \leq \frac{|\Delta \theta|}{2} \ \ \text{and} \ \ \bigg| \sin \bigg( \frac{\theta + \Delta \theta}{2} \bigg) \bigg| < \sin \bigg( \frac{\theta}{2} \bigg) + \frac{| \Delta \theta |}{2}
\end{equation*}

Then:

\begin{equation*}
\begin{split}
    \frac{|\Delta M|}{2N}  & < \bigg( 2 \overbrace{\sin \bigg( \frac{\theta}{2} \bigg)}^{= \sqrt{M/2N}} + \overbrace{\frac{| \Delta \theta |}{2}}^{\leq 2^{-m - 1}} \bigg) \overbrace{\frac{| \Delta \theta |}{2}}^{\leq 2^{-m - 1}} \\
    \frac{|\Delta M|}{2N}  & < \bigg( 2 \sqrt{\frac{M}{2N}} + \frac{1}{2^{m+1}} \bigg) \frac{1}{2^{m+1}}
\end{split}
\end{equation*}

\begin{empheq}[box=\tcbhighmath]{equation}
    \label{Eq: Quantum Counting M deviation}
    \frac{|\Delta M|}{2N}  < \bigg( 2 \sqrt{2MN} + N 2^{-(m+1)} \bigg) 2^{-m}
\end{empheq}

As an example, we choose $m = \big\lceil \frac{n}{2} + 1 \big\rceil$ and $\epsilon = \frac{1}{6}$. Then it follows that $t = \big\lceil \frac{n}{2} \big\rceil + 3$, se we will need $\Omega ( \sqrt{N})$ iterations and  oracle calls. Our accuracy is $|\Delta M | < \sqrt{\frac{M}{2}} + \frac{1}{4} = O(\sqrt{M)}$.

This example works also to show if there is a solution on N items, that is wheter $M=0$ or $M \not= 0$. If $M=0$ we will have $| \Delta M | < \frac{1}{4}$, so the algorithm must produce the estimate within probability at least $\frac{5}{6}$. Conversely, if $M \not= 0$ then it is easy to verify that the estimate for $M$ is not equal to $0$ at least $\frac{5}{6}$ times.

Another application is to find a solution to the search problem where $M$ is unknown because it is needed to know $M$ in order to known how many grover iterations are needed for the Grover algorithm. Then we first estimate $\theta$ and $M$ with high accuracy using phase estimation and then use Grover's algorithm using $M$ to know $R$. The angular error is at most $\frac{\pi}{4} \bigg( 1 + \frac{|\Delta \theta |}{\theta} \bigg)$ so choosing $m = \lceil n/2 \rceil + 1$ gives an angular error of at least $\frac{\pi}{4} \frac{3}{2}$ this corresponds to a probability of at least $\cos^2 \big( \frac{3 \pi}{8} \big) \approx 0.15$ for the search algorithm. If the probability is just as the former case of $\frac{5}{6}$, then the probability of obtaining a solution for the search problem is $\frac{5}{6} \cos^2 \big( \frac{3 \pi}{8} \big) \approx 0.12$ which can be boosted by the combined counting-search procedure.