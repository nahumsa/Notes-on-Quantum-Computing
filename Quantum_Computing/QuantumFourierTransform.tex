\subsection{Quantum Fourier Transform}
\label{subsec: QFT}

Fourier transform is a very important tool to signal processing. Here we will build a quantum version of a Discrete Fourier Transform that will be used in many quantum algorithms.

\subsubsection{Discrete Fourier Transform}
\label{Subsubsec: DFT}

Discrete fourier transform is the map between two strings $F: (x_0, x_1, \dots, x_{N-1}) \rightarrow ((y_0, y_1, \dots, y_{N-1}))$.

\begin{equation}
F(x_k) = y_k = \frac{1}{\sqrt N} \sum_{l=0}^{N-1} x_l \omega_N^{lk}
\end{equation}

Where: $\omega_N^{lk} =  exp\left(2\pi i \frac{lk}{N}\right)$

\subsubsection{Quantum Fourier Transform}
\label{Subsubsec: QFT}
Quantum Fourier transform does the same thing, but using quantum states: $F:\sum_{i=0}^{N-1} x_i\left| i \right> \rightarrow \sum_{i=0}^{N-1} y_i\left| i \right> $

\begin{equation}
F(x_k) = y_k = \frac{1}{\sqrt N} \sum_{l=0}^{N-1} x_l \omega_N^{lk}
\end{equation}

Where: $\omega_N^{lk} =  exp\left(2\pi i \frac{lk}{N}\right)$

This can be represented by an unitary matrix:
\begin{equation}
F = \frac{1}{\sqrt N} \sum_{x=0}^{N-1}\sum_{y=0}^{N-1} \omega_N^{lk}  \left| y \right> \left< x \right|
\end{equation}
To find the circuit representation of the Fourier transform we need to see how it works on $2^N$ qubits first:

\begin{equation}
\begin{split}
F\left(\left| m \right> _n\right) & = \frac{1}{\sqrt N} \sum_{k=0}^{N-1} \omega_N^{km} \left| k \right>_n \\
  & =  \frac{1}{\sqrt N} \sum_{k=0}^{N-1} exp\left( i \frac{2\pi}{N} mk \right) \left| k \right>_n  \\
  & =  \frac{1}{\sqrt N} \sum_{k_{n-1}=0}^{1} \dots \sum_{k_{0}=0}^{1} exp\left( i \frac{2\pi}{N} m \sum_{l=1}^n \frac{k_{n-l}}{2^l} \right) \left| k_{n-1} \dots k_0 \right>_n  \\
  & =  \frac{1}{\sqrt N} \sum_{k_{n-1}=0}^{1} \dots \sum_{k_{0}=0}^{1} \bigotimes_{l=1}^n exp\left( i \frac{2\pi}{N} m  \frac{k_{n-l}}{2^l} \right) \left| k_{n-l}\right>_n \\ 
  & =  \frac{1}{\sqrt N} \bigotimes_{l=1}^n \left[ \left| 0 \right> + exp\left( i \frac{2\pi m}{2^l} \right) \left| 1\right> \right]
\end{split}
\end{equation}
Now we use the binary representation of $\frac{m}{2^l}$: 
\begin{equation}
\begin{split}
\frac{m}{2^l} & = \sum_{p=1}^n m_{n-p} 2^{n-p-l}  = m_{n-1} 2^{n-1-l} + \dots + m_l 2^0 + \dots + m_0 2^{-l}  \equiv m_{n-1} \dots m_l . m_{l-1} \dots m_0 \\
 & = \sum_{p=1}^{n-l} m_{n-p} 2^{n-p-l} + \sum_{p=1}^{l} \frac{m_{l-p}}{2^l}  
\end{split}
\end{equation}

Therefore:

\begin{equation}
	exp\left[ i \frac{2 \pi m}{2^l} \right] = exp\left[ i 2 \pi \sum_{p=1}^{n-l} m_{n-p} 2^{n-p-l} \right] exp\left[i 2 \pi  \sum_{p=1}^{l} \frac{m_{l-p}}{2^l} \right] = exp\left[i 2 \pi  \sum_{p=1}^{l} \frac{m_{l-p}}{2^l} \right]
\end{equation}
Now, we have that:
\begin{equation}
F(\left| m\right>_n) = \frac{1}{\sqrt{N}} \bigotimes_{l=1}^n \left[ \left| 0 \right> +exp\left[i 2 \pi  \sum_{p=1}^{l} \frac{m_{l-p}}{2^l} \right] \left| 1\right> \right]
\end{equation}

Consider first acting on 2 qubits(n=2):

\begin{equation}
\begin{split}
F(\left| m\right>_n) & =  \frac{1}{\sqrt{4}} \bigotimes_{l=1}^2 \left[ \left| 0 \right> +exp\left[i 2 \pi  \sum_{p=1}^{l} \frac{m_{l-p}}{2^l} \right] \left| 1\right> \right] \\
 & = \frac{1}{2} \left[ \left| 0 \right> +exp\left[i 2 \pi  0.m_0 \right] \left| 1\right>\right] \otimes \left[ \left| 0 \right> +exp\left[i 2 \pi  0.m_1m_0 \right] \left| 1\right>\right]
\end{split}
\end{equation}

Consider the following gate:

\begin{equation}
\begin{split}
R_k^{(0,1)} \left| m \right> \left| 0 \right> & =   \left| 0 \right> \\
R_k^{(0,1)} \left| m \right> \left| 1 \right> &  =  exp \left[ i 2 \pi \frac{m}{2^k} \right]   \left| 1 \right> 
\end{split}
\end{equation}

We can write the QFT as: 

\begin{equation}
F = SWAP[H^{(0)} R_2^{(0,1)} H^{(1)} \left| m \right>_2]
\end{equation}

Where the SWAP gate changes the order of the qubits. This is represented by the following circuit:

% \[ \Qcircuit @C=2em @R=1.2em {
%      \lstick{\ket{m_0}}   & \qw & \ctrl{1} & \gate{H}   & \qswap & \qw \\
%       \lstick{\ket{m_1}}  & \gate{H} & \gate{R_2}  & \qw & \qswap \qwx & \qw
% }\]


This can be easily generalized for n qubits.