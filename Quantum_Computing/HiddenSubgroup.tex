\subsection{Hidden Subgroup Problem}
\label{subsec: Hidden Subgroup Problem}

\textbf{Problem:} Given a known group $G$ and a function $f: G \rightarrow S$, where $S$ is a some finite set. Suppose $f$ has the property that there exists a subgroup $H \leq G$ such that f is a constant within each coset, and distinct on different cosets $f(g) = f(g')$ iff $gH = g'H$. We need to find H.

We assume $f$ can be computed efficiently (in polynomial time in $\log |G|$. Since $H$ may be large, "Finding $H$" typically means finding a generating set for $H$.

It is interesting to note that such an abstract problem has many useful examples that are mapped onto this problem:

\begin{itemize}
    \item \textbf{Simon's Problem:} Here $G$ is the additive group $\mathbb{Z}_2^n = \{ 0 , 1 \}^n$ of size $2^n$, $H = \{ 0, s \}$ for a "hidden" $s \in {0,1}^n$, and satisfies $f(x) = f(y)$ iff $x-y \in H$. Finding $s$ solves Simon's problem.
    
    \item \textbf{Period Finding:} Given an $x$ that is coprime to $N$ and associated function $f : \mathbb{Z} \rightarrow \mathbb{Z}_N^*$ by $f(a) \equiv x^a \mod N$, find the period $r$ of $f$. Since $\langle x \rangle$ is the size-r subgroup of the group $\mathbb{Z}_N^*$, the period $r$ divides $|\mathbb{Z}_N^*| = \phi(N)$ [Lagrange's Theorem]. Hence we can restrict the domain of $f$ to $\mathbb{Z}_{\phi(N)}^*$. This problem is an instance of the HSP:
    
    Let $G=\mathbb{Z}_{\phi(N)}$ and consider its subgroup $H = \langle r \rangle$ of all multiples of $r$ up to $\phi(N)$ [$H=r \mathbb{Z}_{\phi(n)} = \{0, r, 2r, \dots, \phi(N) - r \}$]. Note that becaus of its periodicity, $f$ is constant on each coset $s + H$ and distinct on different cosets. Also, $f$ is efficiently computable by repeated squaring. Since $H$ is generated by $r$, finding the generator of $H$ solves the period-finding problem.
\end{itemize}

We will construct an efficient quantum algorithm if the group is Abelian, but first we need to learn some representation theory.

\subsubsection{Representation Theory}
\label{Subsubsec: Representation Theory}

The idea is to replace group elements by matrices in order to use linear algebra in group theory.

\begin{definition}[Representation of a Group]
A $d$-dimensional representation of an multiplicative group $G$ is a map $f: g \mapsto \rho(g)$ from $G$ to the set of all $dxd$ invertible complex matrices, satisfying: $\rho(gh) = \rho(g) \rho(h)$ , $\forall g,h \in G$.
\end{definition}

The latter property makes $rho$ a homomorphism. The \textit{character} corresponding to $\rho$ is the map $\chi_\rho : G \rightarrow \mathbb{C}$ defined by $\chi_\rho (g) = \text{Tr} \big( \rho(g) \big)$.

Let's now consider $G$ abelian. We may consider $d=1$ without loss of generality, then $\rho$ and $\chi_\rho$ are the same function. THe complex values of $\chi_\rho(g)$ have modulus 1, because $| \chi(g^k)| = |\chi(g)|^k$, $\forall k \in \mathbb{Z}$.


\begin{theorem}[Basis Theorem]
Every finite abelian group $G$ is isomorphic to a direct product $\mathbb{Z}_{N_1} \times \dots \times \mathbb{Z}_{N_k}$ of cyclic groups.
\end{theorem}

Let's consider just one cyclic group $(\mathbb{Z}_N, +)$. The discrete fourier transform is an $N \times N$ matrix, ignoring the normalizing factor the $k$-th column may be viewed as a map $\chi_k: \mathbb{Z}_N \rightarrow \mathbb{C}$ defined by $\chi_k(j) = \omega_N^{jk}$, where $\omega_N = e^{\frac{2 \pi i}{N}}$. Let's check if it is a representation of $\mathbb{Z}_N$:

\begin{equation*}
\begin{split}
    \chi_k (j + j') & = \omega_N^{(j + j')k} = e^{\frac{2 \pi i }{N} [(j + j') k]} =  e^{\frac{2 \pi i j k }{N}} e^{\frac{2 \pi i j' k }{N}} \\
    & = \omega_N^{jk}  \omega_N^{j'k} = \chi_k(j) \chi_k(j')
\end{split}
\end{equation*}

In fact, the N characters corresponding to N columns of the Fourier matrix are all characters of $\mathbb{Z}_N$. For abelian groups $G$ that are (isomorphic to) a product $\mathbb{Z}_{N_1} \times \dots \mathbb{Z}_{N_k}$ of cyclic groups, the $|G| = N_1 \times \dots N_k$ characters are just the product of the characters of the individual cyclic groups $\mathbb{Z}_{N_j}$.

The set of all characters of $G$ forms a group $\hat{G}$ with operations of pointwise multiplication. This is called the dual group of $G$. If $H \leq G$, then the following is a sugroup of $\hat{G}$ of size $\frac{|G|}{|H|}$:

\begin{equation*}
    H^\perp = \{ \ \chi_k \ | \ \chi_k(h) = 1 \ \forall h \in H \}
\end{equation*}

We can now interpret the QFT in terms of characters. For $k \in \mathbb{Z}_N$:

\begin{equation*}
    \ket{\chi_k} = \frac{1}{\sqrt{N}} \sum_{j=0}^{N-1} \chi_k \ket{j} = \frac{1}{\sqrt{N}} \sum_{j=0}^{N-1} \omega^{jk}_N \ket{j}
\end{equation*}

Then QFT maps $\mathbb{C}^N$ to the orthonormal basis of characters:

\begin{equation*}
    F_N : \ket{k} \mapsto \ket{\chi_k}
\end{equation*}

\subsubsection{Algorithm for Hidden Subgroup}
\label{Subsub: Algorithm hidden subgroup}
Now we can construct the algorithm for the hidden subgroup problem:

\begin{enumerate}
    \item Start with $\ket{\psi} = \ket{0}\ket{0}$, where the two registers have $|G|$ and $|S|$ respectively.
    \item Create a uniform superposition over $G$ in the first register: 
    \begin{equation*}
        \ket{\psi} = \frac{1}{\sqrt{|G|}} \sum_{g \in G} \ket{g} \ket{0}
    \end{equation*}
    
    \item Compute $f$ on superposition, with $f: G \rightarrow S$:
    \begin{equation*}
        \ket{\psi} = \frac{1}{\sqrt{G}} \sum_{g \in G} \ket{g} \ket{f(g)}
    \end{equation*}
    
    \item Measure the 2nd register. THis yields some value $f(s)$ for unknown $s \in G$. The 1st register collapses to a superposition over $g$ with the same $f$-value as s (i.e. the coset $s + H$):
    \begin{equation*}
        \ket{\psi} = \frac{1}{\sqrt{H}} \sum_{h \in H} \ket{s + h}
    \end{equation*}
    
    \item Apply the "QFT" corresponding to $G$:
    \begin{equation*}
        \ket{\psi} = \frac{1}{\sqrt{H}} \sum_{h \in H} \ket{\chi_{s+h}}
    \end{equation*}
    
    \item Measure and output the resulting $g$.
\end{enumerate}

The key ingredient of this algorithm is that step 5 maps the uniform superposition over the cosets $s + H$ to a uniform superposition over the labels of $H^\perp$:

\begin{equation*}
\begin{split}
    \frac{1}{\sqrt{H}} \sum_{h \in H} \ket{\chi_{s+h}} & = \frac{1}{\sqrt{|H| |G|}} \sum_{h \in H} \sum_{g \in G} \chi_{s+h} (g) \ket{g} \\
    & = \frac{1}{\sqrt{|H||G|}} \sum_{g \in G} \chi_s(g) \sum_{h \in H} \chi_h (g) \ket{g} \\
    & = \sqrt{\frac{|H|}{|G|}} \sum_{g: \chi_g \in H^\perp} \chi_s(g) \ket{g}
\end{split}
\end{equation*}

The last equality follows from the orthogonality of the group $H$ ( $\chi_g$ restricted to $H$ is a character to $H$, and it's the constant-1 character iff $\chi_g \in H^\perp$):

\begin{equation*}
    \sum_{h \in H} \chi_h(g) = \sum_{h \in H} \chi_g(h) = \begin{cases}
    |H| \ , \ \text{if} \ \chi_g \in H^\perp \\
    0 \ , \ \text{if} \ \chi_g \not\in H^\perp \\
    \end{cases}
\end{equation*}

The above algorithm samples uniformly from the elements of $H^\perp$. Each such elements $\chi_g \in H^\perp$ gives us a constraint on $H$ because $\chi_g(h) = 1$ $\forall h \in H$. Generating a small number of such elements will give sufficient information to find the generators of $H$ itself. Let's consider our HSP examples:

\begin{itemize}
    \item \textbf{Simon's Problem:} $G = \mathbb{Z}_2^N = \{ 0, 1 \}^N$  and $H \{ 0, s\}$. Setting up the superposition over $G$ can e done by $H^{\otimes N}$ on the initial state $\ket{0}^{\otimes N}$. The "QFT" corresponding to $G$ is just $H^{\otimes N}$. The $2^N$ character functions are $\chi_g(h) = (-1)^{x \cdot g}$. The algorithm will uniformly sample from labels of elements of:
    \begin{equation*}
        H^\perp = \{ \chi_g \ | \ \chi_g(h) = 1 \ \forall h \in H \} = \{ \chi_g \ | \ g \cdot s = 0 \}
    \end{equation*}
    
    Accordingly, the algorithm samples uniformly from $g \in \{ 0,1 \}^N$ such that $g \cdot s \equiv 0 \mod 2$. Doing this an expected $O(N)$ times gives $n-1$ linearly independent equations about $s$, from which we can find $s$ using Gaussian Elimination.
    
    \item \textbf{Period Finding:} $G = \mathbb{Z}_{\phi(N)}$ and $H = \langle r \rangle $, and:
    \begin{equation*}
        H^\perp = \{ \chi_b \ | \ e^{\frac{2 \pi i b h }{\phi(N)}} = 1 \forall h \in H \} = \{ \chi_b \ | \ \frac{br}{\phi(N)} \in \{0, \dots, r-1 \} \}
    \end{equation*}
    
    Accordingly, the output of the algorithm is an integer multiple $b = \frac{c \phi(N)}{r}$ of $\frac{\phi(N)}{r}$, for uniformly random $c \in \{0, \dots r-1 \}$
\end{itemize}