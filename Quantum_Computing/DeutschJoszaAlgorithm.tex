\subsection{Deutsch-Josza Algorithm}
\label{subsec: DJ}

In the Deutsch's Algorithm we had only 1 bit boolean function, for the Deutsch-Josza case, we have a N bit boolean function $f: \{0,1 \}^N \rightarrow \{0,1 \}$, where f is constant or balanced:

\begin{itemize}
    \item f is constant if $\forall x \in \{0,1 \}^N$, $f(x) = b \in \{0,1 \}$
    \item f is balanced if  $f(x) = b$ for half of the inputs and $f(x) = b \oplus 1$ for the other half.
\end{itemize}

Classicaly, for a deterministic algorithm, it is needed $\frac{2^N}{2} + 1 = 2^{N-1} + 1$ queries.

The Deutsch-Josza Algorithm is represented by the following circuit:

\begin{figure}[H]
    \centering
    \begin{quantikz}
    \lstick{\ket{0}} &  [2mm] \gate{H}\qwbundle{n} & \gate[wires=2][2cm]{U_f} &[2mm]  \qw \qwbundle{n} &  \gate{H} & \meter{}\\
    \lstick{\ket{1}} & \gate{H} &                          & \qw & \qw & \qw
    \end{quantikz}

    \caption{Deutsch-Josza Algorithm, the slash on the circuit represent N qubits.}
    \label{fig: DJ Algorithm Circuit}
\end{figure}

Observe that:

For 1 qubit:
\begin{center}
\begin{quantikz}
\lstick{\ket{0}} & \gate{H} & \qw{\rstick{$\frac{\ket{0} + \ket{1}}{\sqrt{2}}$}}
\end{quantikz}
\end{center}


For 2 qubits:

\begin{center}
\begin{quantikz}
\lstick{\ket{0}} & \gate{H} & \rstick{$\frac{\ket{0} + \ket{1}}{\sqrt{2}}$} \qw \\
\lstick{\ket{0}} & \gate{H} & \rstick{$\frac{\ket{0} + \ket{1}}{\sqrt{2}}$} \qw
\end{quantikz}
\end{center}

Therefore, for N qubits we have a combination of all possible strings:

\begin{equation}
    \ket{\psi} = \frac{1}{\sqrt{2^N}} \sum_{x=0}^{2^N -1} \ket{x}
\end{equation}

So in the first step of the algorithm, we have:

\begin{equation}
    \ket{\psi_0} = \ket{0}^{\otimes N} \ket{1}
\end{equation}

After the Hadamard gates:

\begin{equation}
    \ket{\psi_1} = H^{\otimes N+1} \ket{\psi_0} = \frac{1}{2^{N/2}} \sum_{x=0}^{2^N -1} \ket{x} \otimes \bigg( \frac{ \ket{0} + \ket{1}}{\sqrt{2}} \bigg)
\end{equation}

In the next step we use the oracle as on the Deutsch's Algorithm, since the oracle must be linear, we only need to check one term of the sum:

\begin{equation}
    U_f \ket{x}\bigg( \frac{ \ket{0} + \ket{1}}{\sqrt{2}} \bigg) = \frac{\ket{x}\ket{0 \oplus f(x)} - \ket{x}\ket{1 \oplus f(x)}}{\sqrt{2}}
\end{equation}

We have two cases:
\begin{itemize}
    \item If $f(x)=0$, then:

\begin{equation}
    U_f \ket{x}\bigg( \frac{ \ket{0} + \ket{1}}{\sqrt{2}} \bigg) =  \ket{x}\bigg( \frac{ \ket{0} - \ket{1}}{\sqrt{2}} \bigg) 
\end{equation}

    \item If $f(x)=1$, then:

\begin{equation}
    U_f \ket{x}\bigg( \frac{ \ket{0} + \ket{1}}{\sqrt{2}} \bigg) =  \ket{x}\bigg( \frac{- \ket{0} + \ket{1}}{\sqrt{2}} \bigg) = - \ket{x}\bigg( \frac{ \ket{0} - \ket{1}}{\sqrt{2}} \bigg)
\end{equation}
\end{itemize}

We can represent both cases by:

\begin{equation}
    U_f \ket{x}\bigg( \frac{ \ket{0} + \ket{1}}{\sqrt{2}} \bigg) = (-1)^{f(x)} \ket{x}\bigg( \frac{ \ket{0} - \ket{1}}{\sqrt{2}} \bigg)
\end{equation}

Therefore, after the Oracle:

\begin{equation}
    \ket{\psi_2} = U_f \ket{\psi_1} = \frac{1}{2^{N/2}} \sum_{x=0}^{2^N -1} (-1)^{f(x)}  \ket{x} \otimes \bigg( \frac{ \ket{0} - \ket{1}}{\sqrt{2}} \bigg)
\end{equation}

Before the next step, observe that:

\begin{equation}
\begin{split}
    H\ket{0} = \frac{(-1)^{0\cdot0}\ket{0} + (-1)^{0 \cdot 1}\ket{0}}{\sqrt{2}} \\
    H\ket{1} = \frac{(-1)^{1 \cdot0} \ket{0} + (-1)^{1\cdot1}\ket{0}}{\sqrt{2}} \\
\end{split}
\end{equation}

Therefore:
\begin{equation}
    H\ket{b} = \sum_{b'=0}^{1}\frac{(-1)^{b \odot b'} \ket{b'} }{\sqrt{2}}
\end{equation}

Where $\odot$ is the bitwise sum.

Now we can apply the Haddamard gate to all N qubits:
\begin{equation}
    \begin{split}
        \ket{\psi_3} = & H^{\otimes N} \otimes \mathds{1} \ket{\psi_2} = \frac{1}{2^{N/2}} \sum_{x_1,\dots, x_N=0}^{1} (-1)^{f(x)}  H\ket{x_1} \otimes H\ket{x_2} \otimes \dots \otimes H\ket{x_N} \otimes \bigg( \frac{ \ket{0} - \ket{1}}{\sqrt{2}} \bigg) \\
        = & \frac{1}{2^{N/2}} \sum_{x=0}^{2^N -1} (-1)^{f(x)}  \bigg( \sum_{z_1=0}^1 \frac{(-1)^{z_1 \cdot x_1} \ket{z_1}}{\sqrt{2}} \bigg) \otimes \bigg( \sum_{z_2=0}^1 \frac{(-1)^{z_2 \cdot x_2} \ket{z_2}}{\sqrt{2}} \bigg) \otimes \dots \otimes \bigg( \sum_{z_N=0}^1 \frac{(-1)^{z_N \cdot x_N} \ket{z_N}}{\sqrt{2}} \bigg) \otimes \bigg( \frac{ \ket{0} - \ket{1}}{\sqrt{2}} \bigg) \\
        = & \frac{1}{2^{N}} \sum_{x=0}^{2^N -1} (-1)^{f(x)}  \sum_{z=0}^{2^N - 1} (-1)^{z_1 \cdot x_1 + \dots + z_N \cdot x_N} \ket{z_1 \dots z_N} \otimes \bigg( \frac{ \ket{0} - \ket{1}}{\sqrt{2}} \bigg) \\
    \end{split}
\end{equation}

Therefore:

\begin{equation}
    \ket{\psi_3} = \frac{1}{2^{N}} \sum_{x=0}^{2^N -1} \sum_{z=0}^{2^N - 1} (-1)^{f(x)}   (-1)^{z \odot x} \ket{z} \otimes \bigg( \frac{ \ket{0} - \ket{1}}{\sqrt{2}} \bigg)
\end{equation}

The probability to measure $\ket{00\dots00}$ is given by: 

\begin{equation}
    Pr(00 \dots 00) = \Bigg| \frac{1}{2^N} \sum_{x=0}^{2^N-1} (-1)^{f(x)} \Bigg|^2
\end{equation}

We have two cases as before:

\begin{itemize}
    \item If f is constant:
    
    \begin{equation}
        \sum_{x=0}^{2^N-1} (-1)^{f(x)} = (-1)^{f(x)} 2^N
    \end{equation}
     
    Therefore: $Pr(00\dots00) = 1$
    
    \item If f is balanced:
    
    \begin{equation}
        \sum_{x=0}^{2^N-1} (-1)^{f(x)} = \frac{1}{2} \sum_{x=0}^{(2^N-1)/2} 1 - \frac{1}{2} \sum_{x=0}^{(2^N-1)/2} 1 = 0
    \end{equation}
    
    Therefore: $Pr(00\dots00) = 0$
    
\end{itemize}

So we know that the function is constant if we measure $\ket{00\dots00}$, any other result the function is balanced. Since with only one query we can know if the function is constant or balanced, we have an exponential gain compared with the classical deterministic case.